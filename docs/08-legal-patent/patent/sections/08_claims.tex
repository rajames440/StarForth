% ===========================================
% 08_claims.tex
% Claims
% ===========================================

\section{Claims}

\noindent\textbf{Claim 1 (Independent – System).}  
A computer-implemented virtual machine system comprising:  
(a) a runtime state vector storing values representing real-time execution
characteristics of a workload, the values including at least execution heat,
entropy, and a measure of stability;  
(b) a plurality of feedback loops configured to modify internal runtime
parameters in response to the values of the runtime state vector;  
(c) a set of two or more execution modes, each execution mode defining a
respective configuration of the feedback loops; and  
(d) a supervisory mode selector configured to select one of the execution
modes based on the runtime state vector and to control transitions between
execution modes in a bounded, non-oscillatory manner;  
wherein the virtual machine adjusts its runtime behavior during program
execution by selecting among the execution modes in response to the observed
workload.

\vspace{1em}

\noindent\textbf{Claim 2 (Independent – Method).}  
A method for adaptive execution of a program in a virtual machine, the method
comprising:  
(a) computing a runtime state vector including values that characterize current
workload behavior;  
(b) updating one or more feedback loops using the runtime state vector;  
(c) classifying the workload into at least one behavioral category selected
from stable, temporal, volatile, or transitional;  
(d) selecting, by a supervisory controller, an execution mode from among a
plurality of execution modes based on the runtime state vector and the
classification; and  
(e) adjusting internal interpreter or runtime parameters according to the
selected execution mode;  
whereby the virtual machine achieves stable and optimized performance across
changing workload conditions.

\vspace{1em}

\noindent\textbf{Claim 3 (Independent – Shape-Invariant Operation).}  
A computer-implemented system for achieving shape-invariant runtime performance
comprising:  
(a) a measurement subsystem configured to compute entropy, variance, and
execution heat of a workload;  
(b) a mode-selection subsystem configured to select one of multiple execution
modes based on the measurements; and  
(c) a stabilization subsystem configured to regulate transitions between the
execution modes such that performance variance remains below a threshold for a
plurality of waveform families;  
wherein the system maintains substantially consistent performance across
distinct input waveforms.

\vspace{1.5em}

% -------------------------------
% DEPENDENT CLAIMS
% -------------------------------

\noindent\textbf{Claim 4.}  
The system of Claim 1, wherein the runtime state vector further comprises
pipeline pressure, lookup latency, cache hit rate, or a decay parameter.
\par\medskip
\noindent\textbf{Claim 5.}  
The system of Claim 1, wherein at least one feedback loop modifies a decay rate
for execution heat based on temporal characteristics of the workload.
\par\medskip
\noindent\textbf{Claim 6.}  
The system of Claim 1, wherein one feedback loop performs statistical inference
on historical measurements to generate predictive indicators used in mode
selection.
\par\medskip
\noindent\textbf{Claim 7.}  
The system of Claim 1, wherein the supervisory mode selector applies hysteresis
thresholds to prevent oscillatory transitions between execution modes.
\par\medskip
\noindent\textbf{Claim 8.}  
The system of Claim 1, wherein the execution modes comprise at least a baseline
mode, a temporal mode, an inference-driven mode, and a fully adaptive mode.
\par\medskip
\noindent\textbf{Claim 9.}  
The system of Claim 1, wherein the virtual machine converges to a stable
steady-state configuration when variance of recent execution timings falls
below a threshold.
\par\medskip
\noindent\textbf{Claim 10.}  
The system of Claim 2, wherein classifying the workload comprises evaluating an
entropy window representing distributional characteristics of execution heat.
\par\medskip
\noindent\textbf{Claim 11.}  
The system of Claim 2, wherein execution heat is increased upon invocation of a
word or instruction and decays over time according to a selected decay function.
\par\medskip
\noindent\textbf{Claim 12.}  
The system of Claim 2, wherein selecting the execution mode comprises comparing
the runtime state vector to one or more pre-validated configuration profiles.
\par\medskip
\noindent\textbf{Claim 13.}  
The system of Claim 3, wherein waveform families comprise at least sinusoidal,
square-wave, triangular, burst-like, or mixed-input patterns.

\noindent\textbf{Claim 14.}  
The system of Claim 3, wherein performance consistency is measured using
coefficient of variation.
\par\medskip
\noindent\textbf{Claim 15.}  
The system of Claim 1, wherein feedback loops collectively modify lookup
strategies of the virtual machine.
\par\medskip
\noindent\textbf{Claim 16.}  
The system of Claim 1, wherein the supervisory mode selector prevents mode
transitions until a confidence value computed from the runtime state vector
exceeds a threshold.
\par\medskip
\noindent\textbf{Claim 17.}  
The method of Claim 2, wherein adjusting internal runtime parameters comprises
modifying caching behavior, prefetching behavior, or traversal logic of a
dictionary-based interpreter.
\par\medskip
\noindent\textbf{Claim 18.}  
The method of Claim 2, wherein the workload classification employs rolling
measurements or sliding windows.
\par\medskip
\noindent\textbf{Claim 19.}  
The system of Claim 1, wherein the virtual machine comprises a stack-based
interpreter, a threaded-code interpreter, a just-in-time compilation
environment, an embedded controller, or a microkernel component.
\par\medskip
\noindent\textbf{Claim 20.}  
The system of Claim 1, wherein the state vector is maintained independently for
each execution thread in a multi-threaded environment.

\newpage

