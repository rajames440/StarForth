% ===========================================
% 12_appendix.tex
% Appendix (Non-limiting Supplemental Material)
% ===========================================

\section*{Appendix}

This Appendix provides non-limiting supplemental material intended to support
technical understanding of the invention. The contents herein are provided for
clarity and completeness only and shall not be construed as limiting the scope
of the invention or the claims set forth in this application.

\subsection{Terminology Glossary}

\textbf{Execution Heat} — A scalar value representing recency and frequency of
invoked instructions.

\textbf{Entropy Window} — A rolling distribution describing variability of
execution heat over a selected interval.

\textbf{Pipeline Pressure} — A measure of how heavily the execution pipeline is
utilized under current workload conditions.

\textbf{Stability Score} — A value derived from variance or coefficient of
variation of recent execution timings.

\textbf{Feedback Loop} — A control mechanism governing internal runtime
parameters based on observed signals.

\textbf{Execution Mode} — A pre-validated configuration profile describing a
specific arrangement of active feedback loops.

\textbf{Jacquard Mode Selector (L8)} — Supervisory controller that selects
execution modes based on the runtime state vector.

\textbf{Shape-Invariant Operation} — Maintenance of stable performance metrics
across multiple input waveform families.

\subsection{Experimental Setup (Non-limiting)}

The validation experiments described herein were performed using representative
virtual machine builds. While specific implementations may vary, typical
evaluation conditions included:

\begin{itemize}
	\item factorial sweeps across multiple feedback-loop combinations;
	\item 30-rep and 300-rep test batches for statistical robustness;
	\item comparable workloads across stable, diverse, volatile, temporal, and
	transitional categories;
	\item waveform-based evaluation including sinusoidal, triangular, square-wave,
	burst-pattern, and mixed-pattern sequences;
	\item measurement of mean time per word, variance, coefficient of variation,
	heat distributions, and adaptive mode usage.
\end{itemize}

Representative results are summarized in the figures and tables referenced in
Sections 05 and 09.

\subsection{Representative Mode Configuration Profiles}

The following profiles illustrate typical mode configurations. These examples
are non-exclusive and non-exhaustive.

\textbf{Mode 0 – Baseline:} L7 active; minimal inference weighting.

\textbf{Mode 1 – Temporal:} L3 emphasized; decay rate modulation active; L5
stabilization applied.

\textbf{Mode 2 – Inference:} L2 and L6 emphasized; predictive weighting enabled.

\textbf{Mode 3 – Full Adaptive:} L1–L7 coordinated; dynamic switching thresholds
adjusted based on entropy slope and stability score.

\subsection{Alternate Formulations of the State Vector}

The state vector may be implemented in various forms including:

\begin{itemize}
	\item fixed-length numeric vector;
	\item sliding-window statistics coupled with temporal markers;
	\item probability distribution parameters;
	\item sparse event-driven representation;
	\item hybrid combinations thereof.
\end{itemize}

Any formulation capable of characterizing runtime behavior falls within the
scope of the invention.

\subsection{Supplementary Notes on Stability Analysis}

The system’s convergence behavior may be analyzed using:

\begin{itemize}
	\item coefficient of variation thresholds,
	\item rolling-variance collapse detection,
	\item entropy-slope flattening,
	\item or mode-transition damping curves.
\end{itemize}

These analyses are provided for explanatory purposes only and are not required
for practicing the invention.

\subsection{Representative Workload Families}

The following generalized workload families were used in validation:

\begin{itemize}
	\item \textbf{Stable:} Repetitive inner loops with minimal structural change.
	\item \textbf{Temporal:} Gradually shifting operation sequences.
	\item \textbf{Volatile:} Burst-driven or highly irregular workloads.
	\item \textbf{Transitional:} Workload shifts occurring over short periods.
	\item \textbf{Mixed:} Stochastic combinations of the above patterns.
\end{itemize}

\subsection{Implementation Neutrality Statement}

All examples in this Appendix are illustrative. The invention may be practiced
in any programming language, virtual machine architecture, or hardware-
software integration capable of maintaining a state vector, coordinating
feedback loops, and selecting among runtime modes.

