\begin{abstract}
	\normalsize

	A memristive virtual machine architecture and computational physics system are disclosed that exhibit measurable physical laws, fundamental constants, and emergent phenomena analogous to wave mechanics, thermodynamics, and quantum systems. The virtual machine maintains a runtime state vector representing execution heat, entropy, temporal dynamics, and stability indicators, where execution heat functions as a memristive state variable with history-dependent conductance properties creating hysteresis loops in phase space.
\par\medskip
	The system exhibits standing wave resonance at a natural frequency of 0.6667 cycles per window configuration, with constructive interference enabling quantized state transitions at specific architectural boundaries (6144 bytes, 16384 bytes, 32768 bytes). An intrinsic characteristic length scale of 256 bytes emerges from five independent physical mechanisms: cache line alignment, working set optimization, heat decay timescale, pipelining depth, and dimensional reduction from an 8-degree-of-freedom state space (2^8 = 256).
\par\medskip
	Golden ratio phenomena (φ ≈ 1.618) govern cache interference patterns, with performance penalties of 60\% occurring at window sizes W = φ × 2^N, while Fibonacci-sequence windows 
	naturally avoid these penalties through harmonic alignment. The system implements computational field theory with wave equations governing the evolution of execution parameters, exhibiting Maxwell-equation analogs for heat and performance fields, Lagrangian formulations with conservation laws, and thermodynamic cooling via free energy minimization.
\par\medskip
	Quantum-analog effects include: (1) measurement-induced state collapse where heartbeat observation forces selection between dual attractor states; (2) probabilistic tunneling between locked and escaped regimes with escape probability proportional to resonance amplitude; (3) quantized energy levels where K=1.0 (perfect James Law compliance) achieves exact integer ratios at resonance peaks with 3.3\% probability; and (4) Heisenberg-like timing uncertainty at picosecond scales (Q48.16 fixed-point precision capturing 15.3-picosecond resolution).
\par\medskip
	The James Law of Computational Dynamics is disclosed as the governing equation:
	\[
	K = \frac{\Lambda_{\text{eff}}}{W} \times \left[1 + A(W) \times \sin(2\pi f_0 \log_2(W) + \varphi)\right]
	\]
	where K is the measured statistic ratio, Λ\_eff = 256 bytes is the intrinsic wavelength, W is configured window size, f₀ = 0.6667 cycles/window is natural frequency, A(W) is damped amplitude envelope, and φ is phase offset. This equation accurately predicts system behavior across 360 experimental runs with zero algorithmic variance (entropy = 0.0), validated coefficient of variation below 1\%, and reproducible fundamental constants.
\par\medskip
	The disclosed system autonomously discovers optimal operating points through physical principles rather than heuristic tuning. A supervisory mode selector (Jacquard controller) coordinates seven feedback loops controlling heat tracking, rolling window dynamics, linear decay, pipelining metrics, window inference, decay inference, and adaptive heartbeat timing. Mode selection exploits resonance peaks for enhanced performance, avoids anti-resonance troughs where the system rigidly locks to intrinsic scales, and leverages harmonic relationships (3:2 frequency ratio between K oscillation and performance oscillation creating Lissajous-figure phase space trajectories).
\par\medskip
	Hardware resonance at W=4096 bytes exhibits zero variance across all experimental trials due to triple-lock alignment: page boundary (4KB virtual memory), cache alignment (64 cache lines), and binary quantization (K = 1/16 exactly representable). Escaped-regime runs at 8KB and 16KB windows demonstrate 4-6\% performance improvement over locked-regime runs, indicating architectural favorability at L1→L2 cache transition boundaries.
\par\medskip
	The invention is applicable to stack-based virtual machines, threaded interpreters, just-in-time compilation systems, embedded runtimes, real-time operating systems, microkernel subsystems, neuromorphic computing architectures, and any computational system requiring stable, predictable, self-optimizing behavior with measurable physical properties. By establishing computational physics as a practical engineering discipline, the disclosed architecture enables design of systems with reproducible constants, predictive mathematical frameworks, and emergent optimization through natural laws rather than algorithmic heuristics.

\end{abstract}