\begin{abstract}
	\normalsize
	
	A virtual machine architecture and adaptive execution system are disclosed that
	dynamically modifies internal runtime behavior in response to real-time workload
	characteristics. The system maintains a continuously updated runtime state vector
	representing execution heat, entropy, variance, temporal decay dynamics, pipeline
	pressure, and related stability indicators. A plurality of coordinated feedback
	loops adjust internal parameters including decay functions, inference weights,
	lookup strategies, cache behavior, and prefetch timing. A supervisory controller
	selects among multiple execution modes, each mode defining a distinct and
	validated configuration of feedback-loop activity, and regulates transitions
	using bounded, non-oscillatory mechanisms such as hysteresis thresholds,
	confidence scoring, or convergence constraints.
\par\medskip
	The system characterizes workloads into behavioral families—including stable,
	temporal, volatile, transitional, and mixed patterns—and autonomously selects the
	execution mode appropriate to each condition. This enables the virtual machine to
	maintain consistent performance, reduced variance, and shape-invariant behavior
	across heterogeneous, non-stationary, and waveform-diverse workloads. The
	architecture operates without manual tuning or fixed parameters and is robust to
	sudden workload shifts, burst-like behavior, and long-term temporal drift. The
	invention is applicable to stack-based interpreters, embedded runtimes,
	just-in-time compilation environments, microkernel subsystems, distributed
	execution engines, and real-time or soft–real-time systems requiring predictable
	and stable execution characteristics. The disclosed architecture provides a
	general-purpose method for achieving autonomous optimization, rapid convergence
	to steady-state behavior, and high stability through integrated feedback,
	statistical inference, and supervisory mode selection.
	
\end{abstract}
