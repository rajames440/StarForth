% ===========================================
% 03_background.tex
% Background of the Invention
% ===========================================

\section{Background}

Virtual machines, interpreters, and software execution engines traditionally
operate using a fixed set of internal configuration parameters. These
parameters govern runtime behavior such as lookup mechanisms, caching
strategies, decay functions, and instruction scheduling heuristics. In most
systems, these internal settings are static: they are either hard-coded,
selected at compile-time, or chosen manually by the developer based on
anticipated workloads.
\par\medskip
While such static approaches may provide acceptable performance for narrowly
defined or predictable execution patterns, they suffer significant limitations
in real-world conditions where workloads may vary widely over time. Modern
software environments frequently exhibit heterogeneous and dynamic execution
characteristics, including:

\begin{itemize}
	\item highly repetitive or stable workloads,
	\item gradually shifting temporal patterns,
	\item intermittent bursts of unpredictable activity,
	\item transitions between different workload phases,
	\item and non-stationary sequences that do not conform to a single
	behavioral profile.
\end{itemize}

The inability of static configuration systems to adapt to these diverse
conditions often leads to suboptimal performance, elevated variance, and
instability. In particular, virtual machines with multiple interacting
parameters may exhibit strong sensitivity to the selection of initial settings.
A configuration that performs well under one workload may perform poorly under
another, resulting in a substantial performance spread across the possible
parameter space.
\par\medskip
Efforts to address this problem in existing systems typically rely on either
(1) manual tuning based on domain expertise, or (2) simplistic heuristics that
enable limited runtime adjustment. Manual tuning is labor-intensive, brittle,
and non-generalizable. Heuristic-based adaptation, on the other hand, often lacks
robustness: it may overreact to short-term fluctuations, underreact to sustained
changes, or oscillate between competing strategies. Such approaches generally
fail to maintain predictable or stable behavior across diverse conditions.
\par\medskip
Additionally, most adaptive mechanisms found in prior systems do not incorporate
statistical inference, coarse-grained workload characterization, or coordinated
feedback-loop orchestration. Instead, they adjust isolated parameters in
isolation, without understanding the underlying structure of the workload or the
interdependencies between internal subsystems. As a result, they frequently
introduce instability, variance, or pathological behavior when faced with
non-stationary execution patterns.
\newpage
Furthermore, existing literature and industrial practice provide little
guidance for achieving shape-invariant performance — that is, maintaining
consistent and predictable execution characteristics across different workload
waveforms or input signal shapes. Without such invariance, adaptive systems may
behave unpredictably when presented with novel or diverse execution patterns.
\par\medskip
In summary, the state of the art lacks:

\begin{itemize}
	\item robust, workload-aware adaptation mechanisms,
	\item coordinated feedback-loop control for virtual machine internals,
	\item systems capable of identifying and selecting optimal runtime
	configurations dynamically,
	\item methods for stable, non-oscillatory adaptation,
	\item and provably consistent behavior across heterogeneous workload shapes.
\end{itemize}

These deficiencies motivate the need for a new class of virtual machine
architecture — one that is capable of autonomously characterizing workloads
through statistical analysis, selecting appropriate execution modes via
supervisory control logic, coordinating multiple feedback mechanisms through
a unified state vector, and maintaining stable behavior across diverse workload
characteristics including varying execution patterns, variability levels, and
temporal dynamics.

\newpage
