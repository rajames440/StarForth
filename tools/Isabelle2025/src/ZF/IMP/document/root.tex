\documentclass[a4wide]{article}
\usepackage[T1]{fontenc}
\usepackage{isabelle,isabellesym}
\usepackage{pdfsetup}

\urlstyle{rm}
\renewcommand{\isachardoublequote}{}

% pretty printing for the Com language
%\newcommand{\CMD}[1]{\isatext{\bf\sffamily#1}}
\newcommand{\CMD}[1]{\isatext{\rm\sffamily#1}}
\newcommand{\isasymSKIP}{\CMD{skip}}
\newcommand{\isasymASSN}{\CMD{:=}}
\newcommand{\isasymSEQ}{\CMD{;}}
\newcommand{\isasymIF}{\CMD{if}}
\newcommand{\isasymTHEN}{\CMD{then}}
\newcommand{\isasymELSE}{\CMD{else}}
\newcommand{\isasymWHILE}{\CMD{while}}
\newcommand{\isasymDO}{\CMD{do}}

\addtolength{\hoffset}{-1cm}
\addtolength{\textwidth}{2cm}


\begin{document}

\title{IMP --- A {\tt WHILE}-language and two semantics}
\author{Heiko Loetzbeyer and Robert Sandner}
\maketitle

\parindent 0pt\parskip 0.5ex

\begin{abstract}\noindent
  The formalization of the denotational and operational semantics of a simple
  while-language together with an equivalence proof between the two semantics.
  The whole development essentially formalizes/transcribes chapters 2 and 5 of
  \cite{Winskel}.  A much extended version of this development is found in
  HOL/IMP of the Isabelle distribution.
\end{abstract}

\tableofcontents

\parindent 0pt\parskip 0.5ex
\input{session}

\bibliographystyle{plain}
\bibliography{root}

\end{document}
