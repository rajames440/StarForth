\documentclass[11pt,a4paper]{article}
\usepackage[T1]{fontenc}
\usepackage{tikz}
\usepackage{subfigure}
\usepackage[nohyphen,strings]{underscore}
\usepackage{amsmath}
\usepackage{isabelle,isabellesym}
\usepackage{verbatim}
\usepackage{alltt}
\usepackage{array}

\usepackage{amssymb}

\usepackage{pdfsetup}

\isadroptag{theory}
\isafoldtag{proof}

% urls in roman style, theory text in typewriter
\urlstyle{rm}
\isabellestyle{tt}


\begin{document}

\title{Tutorial to Locales and Locale Interpretation%
\thanks{Published in L.~Lamb\'an, A.~Romero, J.~Rubio, editors, {\em Contribuciones Cient\'{\i}ficas en honor de Mirian Andr\'es.}  Servicio de Publicaciones de la Universidad de La Rioja, Logro\~no, Spain, 2010.  Reproduced by permission.}}
\author{Clemens Ballarin}
\date{}

\maketitle

\begin{abstract}
  Locales are Isabelle's approach for dealing with parametric
  theories.  They have been designed as a module system for a
  theorem prover that can adequately represent the complex
  inter-dependencies between structures found in abstract algebra, but
  have proven fruitful also in other applications --- for example,
  software verification.

  Both design and implementation of locales have evolved considerably
  since Kamm\"uller did his initial experiments.  Today, locales
  are a simple yet powerful extension of the Isar proof language.
  The present tutorial covers all major facilities of locales.  It is
  intended for locale novices; familiarity with Isabelle and Isar is
  presumed.
\end{abstract}

\parindent 0pt\parskip 0.5ex

\input{session}

\bibliographystyle{abbrv}
\bibliography{root}

\end{document}

%%% Local Variables:
%%% mode: latex
%%% TeX-master: t
%%% End:
