% This is LLNCSDOC.TEX the documentation file of
% the LaTeX2e class from Springer-Verlag
% for Lecture Notes in Computer Science, version 2.25
\documentclass{llncs}
\usepackage{llncsdoc}
\usepackage{graphicx}
\usepackage{doc}
\usepackage[dvips,bookmarksopen]{hyperref}
%
\makeatletter\@twosidefalse\@mparswitchfalse\makeatother
%
\begin{document}
\title{Instructions for Using Springer's \texttt{llncs} Class for
Computer Science Proceedings Papers}
%
\subtitle{\texttt{llncs}, Version 2.25, Oct 21, 2024}
%
\author{}
\institute{}

\maketitle \thispagestyle{empty}
\markright{Instructions for Using Springer's \texttt{llncs} Class}
%
\section{Installation}
%
Copy \texttt{llncs.cls} to a directory that is searched by \LaTeX{},
e.g. either your \texttt{texmf} tree or the local work directory with your main
{\LaTeX} file.

%
\section{Working with the \texttt{llncs} Document Class}
%
\subsection{General Information}
The \texttt{llncs} class is an extension of the standard \LaTeX{} \texttt{article}
class. Therefore you may use all \texttt{article} commands in your
manuscript.

If you are already familiar with \LaTeX{}, the \texttt{llncs} class should
not give you any major difficulties. It basically adjusts the layout
to the required standard, defining styles and spacing of headings
and captions and setting the printing area to 122\,mm horizontally
by 193\,mm vertically. To keep the layout consistent, we kindly ask
you to refrain from using any \LaTeX{} or \TeX{} command that
modifies these settings (i.e. \verb|\textheight|, \verb|\vspace|,
\verb|baselinestretch|, etc.). Such manual layout adjustments should
be limited to very exceptional cases.

In addition to defining the general layout, the \texttt{llncs} document class
provides some special commands for typesetting the contribution
header, i.e. title, authors, affiliations, abstract, and additional
metadata. These special commands are described in Sect.~\ref{Sec2}.

For a more detailed description of how to prepare your text,
illustrations, and references, see the {\em Springer Guidelines for
Authors of Proceedings}.


\subsection{How to Use the \texttt{llncs} Document Class}
The \texttt{llncs} class is invoked by replacing \texttt{article} by \texttt{llncs} in
the first line of your \LaTeX{} document:
\begin{verbatim}
\documentclass{llncs}
%
\begin{document}
  <Your contribution>
\end{document}
\end{verbatim}
%
If your file is already coded with \LaTeX{}, you can easily adapt it
to the \texttt{llncs} document class by replacing
\begin{verbatim}
\documentclass{article}
\end{verbatim}
with
\begin{verbatim}
\documentclass{llncs}
\end{verbatim}


\section{How to Code the Header of Your Paper}\label{Sec2}
\label{contbegin}
\subsection{Title}
%
\DescribeMacro{\title} Please code the title of your contribution as
follows:
\begin{verbatim}
\title{<Your contribution title>}
\end{verbatim}
All words in titles should be capitalized except for conjunctions,
prepositions (e.g.\ on, of, by, and, or, but, from, with, without,
under), and definite/indefinite articles (the, a, an), unless they
appear at the beginning. Formula letters are typeset as in the text.
Long titles that run over multiple lines can be wrapped explicitly
with \verb|\\|. Titles have no end punctuation.

Acknowledgements should generally be placed in an unnumbered
subsection at the end of the paper. If you still need to refer to a
support or funding program in a note to the title, you can use
the\DescribeMacro{\thanks} \verb|\thanks| macro inside the title:
\begin{verbatim}
\title{<Your contribution title>\thanks{<granted by x>}}
\end{verbatim}
Please do not use \verb|\thanks| inside \verb|\author| or
\verb|\institute| as footnotes for these elements are not supported
in the online version and will therefore be dropped.

\DescribeMacro{\fnmsep} If you need two or more footnotes please
separate them with \verb|\fnmsep| (i.e. {\itshape f}oot\emph note
\emph mark \emph{sep}arator).

\DescribeMacro{\titlerunning} If a long title does not fit in the
single line of the running head, a warning is generated. You can
specify an abbreviated title for the running head with the command
\begin{verbatim}
\titlerunning{<Your abbreviated contribution title>}
\end{verbatim}
\DescribeMacro{\subtitle} An optional subtitle may also be added:
\begin{verbatim}
\subtitle{<subtitle of your contribution>}
\end{verbatim}

\subsection{Author(s)}
\DescribeMacro{\author} The name(s) of the author(s) are specified
by:
\begin{verbatim}
\author{<author(s) name(s)>}
\end{verbatim}
\DescribeMacro{\and} If there is more than one author, please
separate them by \verb|\and|. This makes sure that correct
punctuation is inserted according to the number of authors.

\DescribeMacro{\inst} Numbers referring to different addresses or
affiliations should be attached to each author with the
\verb|\inst{<number>}| command. If an author is affiliated with
multiple institutions the numbers should be separated by a comma,
for example \verb|\inst{2,3}|.

\DescribeMacro{\orcidID} ORCID identifiers can be included with
\begin{verbatim}
\orcidID{<ORCID identifier>}
\end{verbatim}
The ORCID (Open Researcher and Contributor ID) registry provides
authors with unique digital identifiers that distinguish them from
other researchers and help them link their research activities to
these identifiers. Authors who are not yet registered with ORCID are
encouraged to apply for an individual ORCID id at
\url{https://www.orcid.org} and to include it in their papers. In
the final publication, the ORCID id will be replaced by an ORCID
icon, which will link from the eBook to the
actual ID in the ORCID database. The ORCID icon will also replace
the number in the printed book.

If you have done this correctly, the author line now reads, for
example:
\begin{verbatim}
\author{First Author\inst{1}\orcidID{0000-1111-2222-3333} \and
Second Author\inst{2,3}\orcidID{1111-2222-3333-4444}}
\end{verbatim}
The given name(s) should always be followed by the family name(s).
Authors who have more than one family name should indicate which
part of their name represents the family name(s), for example by
non-breaking spaces \verb|Jos\'{e} Martinez~Perez| or curly braces
\verb|Jos\'{e} {Martinez Perez}|.

\DescribeMacro{\authorrunning} As given name(s) are to be shortened
to initials in the running heads, specifying an abbreviated author
list with the optional command:
\begin{verbatim}
\authorrunning{<abbreviated author list>}
\end{verbatim}
might add some clarity about the correct representation of author
names, in the running-heads as well as in the author index.

\subsection{Affiliations}
\DescribeMacro{\institute} Addresses of institutes, companies, etc.
should be given in \verb|\institute|.

\DescribeMacro{\and} Multiple affiliations are separated by
\verb|\and|, which automatically assures correct numbering:
\begin{verbatim}
\institute{<name of an institute>
\and <name of the next institute>
\and <name of the next institute>}
\end{verbatim}
\DescribeMacro{\email}Inside \verb|\institute| you can use
\begin{verbatim}
\email{<email address>}
\end{verbatim}
\DescribeMacro{\url}and
\begin{verbatim}
\url{<url>}
\end{verbatim}
to provide author email addresses and Web pages. If you need to
typeset the tilde character -- e.g. for your Web page in your unix
system's home directory -- the \verb|\homedir| command will do this.
If multiple authors have the same affiliation, please check that the
order of email addresses matches the sequence of (affiliated) author
names.

Please note that, if email addresses are given in your paper, they
will also be included in the metadata of the online version.

\subsection{Format the Header}
\DescribeMacro{\maketitle} The command \verb|\maketitle| formats the
header of your paper. If you leave it out the work done so far will
produce \emph{no} text.

\subsection{Abstract and Keywords}
\DescribeEnv{abstract}The abstract is coded as follows:
\begin{verbatim}
\begin{abstract}
<Text of the summary of your paper>
\end{abstract}
\end{verbatim}
\DescribeMacro{\keywords} Keywords should be specified inside the
\verb|abstract| environment. \DescribeMacro{\and}Please capitalize
the first letter of each keyword and again separate them with
\verb|\and|:
\begin{verbatim}
\keywords{First keyword \and Second keyword \and Third keyword}
\end{verbatim}
The keyword separator will then be properly rendered as a middle
dot.

\section{How to Code the Body of Your Paper}
%
\subsection{General Rules}
From a technical point of view, the \texttt{llncs} document class does not
require any specific {\LaTeX} coding in the body of your paper. You
can simply use the commands provided by the `article' document
class. For more information about what will be done with your
manuscript before publication, please refer to the {\em Springer
Guidelines for Authors of Proceedings}.

\subsection{Special Math Characters}
The \texttt{llncs} document class supports some additional special
characters:\smallskip
\begin{center}
\begin{tabular}{l@{\hspace{1em}yields\hspace{1em}}
c@{\hspace{3em}}l@{\hspace{1em}yields\hspace{1em}}c}
\verb|\grole| & $\grole$ & \verb|\getsto| & $\getsto$\\
\verb|\lid|   & $\lid$   & \verb|\gid|    & $\gid$
\end{tabular}\smallskip
\end{center}
If you need blackboard bold characters, i.e. for sets of numbers,
please load the related \AmSTeX fonts. If for some reason this is
not possible you can also use the following commands from the \texttt{llncs}
class:\smallskip
\begin{center}
\begin{tabular}{l@{\hspace{1em}yields\hspace{1em}}
c@{\hspace{3em}}l@{\hspace{1em}yields\hspace{1em}}c}
\verb|\bbbc| & $\bbbc$ & \verb|\bbbf| & $\bbbf$\\
\verb|\bbbh| & $\bbbh$ & \verb|\bbbk| & $\bbbk$\\
\verb|\bbbm| & $\bbbm$ & \verb|\bbbn| & $\bbbn$\\
\verb|\bbbp| & $\bbbp$ & \verb|\bbbq| & $\bbbq$\\
\verb|\bbbr| & $\bbbr$ & \verb|\bbbs| & $\bbbs$\\
\verb|\bbbt| & $\bbbt$ & \verb|\bbbz| & $\bbbz$\\
\verb|\bbbone| & $\bbbone$
\end{tabular}\smallskip
\end{center}
Please note that all these characters are only available in math
mode.

\section{Theorems, Definitions, and Proofs}\label{builtintheo}
\subsection{Predefined Theorem-Like Environments}
\DescribeEnv{corollary}\DescribeEnv{definition}\DescribeEnv{lemma}\DescribeEnv{proposition}\DescribeEnv{theorem}%
Several theorem-like environments are predefined in the \texttt{llncs}
document class. The following environments have a bold run-in
heading, while the following text is in italics:
\begin{verbatim}
\begin{corollary}   <text> \end{corollary}
\begin{definition}  <text> \end{definition}
\begin{lemma}       <text> \end{lemma}
\begin{proposition} <text> \end{proposition}
\begin{theorem}     <text> \end{theorem}
\end{verbatim}
\DescribeEnv{case}\DescribeEnv{conjecture}\DescribeEnv{example}\DescribeEnv{exercise}\DescribeEnv{note}%
\DescribeEnv{problem}\DescribeEnv{property}\DescribeEnv{question}\DescribeEnv{remark}\DescribeEnv{solution}%
Other theorem-like environments render the text in roman, while the
run-in heading is bold as well:
\begin{verbatim}
\begin{case}        <text> \end{case}
\begin{conjecture}  <text> \end{conjecture}
\begin{example}     <text> \end{example}
\begin{exercise}    <text> \end{exercise}
\begin{note}        <text> \end{note}
\begin{problem}     <text> \end{problem}
\begin{property}    <text> \end{property}
\begin{question}    <text> \end{question}
\begin{remark}      <text> \end{remark}
\begin{solution}    <text> \end{solution}
\end{verbatim}
\DescribeEnv{claim}\DescribeEnv{proof} Finally, there are also two
unnumbered environments that have the run-in heading in italics and
the text in upright roman.
\begin{verbatim}
\begin{claim}       <text> \end{claim}
\begin{proof}       <text> \end{proof}
\end{verbatim}
\DescribeMacro{\qed}Proofs may contain an eye catching square, which
can be inserted with \verb|\qed|) before the environment ends.

\subsection{User-Defined Theorem-Like Environments}
\DescribeMacro{\spnewtheorem}We have enhanced the standard
\verb|\newtheorem| command and slightly changed its syntax to get
two new commands \verb|\spnewtheorem| and \verb|\spnewtheorem*| that
now can be used to define additional environments. They require two
additional arguments, namely the font style of the label and the
font style of the text of the new environment:
\begin{verbatim}
\spnewtheorem{<env_nam>}[<num_like>]{<caption>}{<cap_font>}{<body_font>}
\end{verbatim}
For example,
\begin{verbatim}
\spnewtheorem{maintheorem}[theorem]{Main Theorem}{\bfseries}{\itshape}
\end{verbatim}
will create a {\em main theorem\/} environment that is numbered
together with the predefined {\em theorem\/}. The sharing of the
default counter (\verb|[theorem]|) is desired. If you omit the
optional second argument of \verb|\spnewtheorem|, a separate counter
for your new environment is used throughout your document.

In combination with the (obsolete) class option \verb|envcountsect|
(see. Sect.~\ref{SecClassOptions}), the \verb|\spnewtheorem| command
also supports the syntax:
\begin{verbatim}
\spnewtheorem{<env_nam>}{<caption>}[<within>]{<cap_font>}{<body_font>}
\end{verbatim}
With the parameter \verb|<within>|, you can control the sectioning
element that resets the theorem counters. If you specify, for
example, \verb|subsection|, the newly defined environment is
numbered subsectionwise.

\DescribeMacro{\spnewtheorem*}If you wish to add an unnumbered
environment, please use the syntax
\begin{verbatim}
\spnewtheorem*{<env_nam>}{<caption>}{<cap_font>}{<body_font>}
\end{verbatim}

%
\section{Credits and Acknowledgments}
\label{credits}
\DescribeEnv{credits}
Credits and acknowledgments should be placed at the end of the paper,
just before the references. Please use the \verb|credits| environment
to make sure that both text and run-in headings are printed in small
font size.

\DescribeMacro{\ackname}\DescribeMacro{\discintname}
There are two possible \verb|\subsubsection| headings to provide such
statements: ``Acknowledgments'' and ``Disclosure of Interests''.
The \verb|\ackname| and \verb|\discintname| macros are used to generate
the correct run-in titles.

General acknowledgments can be provided in the (optional) paragraph
``Acknowledgments'', which is followed by the (mandatory) paragraph
``Disclosure of Interests'', where any competing interests of the authors
are to be declared.


%
\section{References}
\label{refer}
%
There are three options for citing references:
\begin{itemize}
\item arabic numbers, i.e. [1], [3--5], [4--6,9],
\item labels, i.e. [CE1], [AB1,XY2],
\item author/year system, (Smith et al. 2000), (Miller 1999a, 12; Brown
2018).
\end{itemize}
We prefer citations with arabic numbers, i.e. the usage of
\verb|\bibitem| without an optional parameter.
\DescribeMacro{citeauthoryear}If you want to use the author/year
system, you can use the class option \verb|citeauthoryear|, i.e.
\begin{verbatim}
\documentclass[citeauthoryear]{llncs}
\end{verbatim}
Please note that this option does not automatically change your
citations to the author/year style. It basically redefines the
\verb|\bibitem| command to take the publication year as an optional
parameter that is displayed instead of an arabic number. Author
name(s) and, if necessary, parentheses are to be typed manually. If
your reference reads
\begin{verbatim}
\bibitem[2016]{vdaalst:2016}
van der Aalst, W.: Process Mining, 2nd ed. Springer, Heidelberg (2016)
\end{verbatim}
and is cited as follows:
\begin{verbatim}
... is shown by van der Aalst (\cite{vdaalst:2016})
\end{verbatim}
the resulting text will be:\begin{quote} ``\dots is shown by van der
Aalst (2016).''
\end{quote}

\DescribeMacro{splncs04.bst}We encourage you to use {\sc Bib}\TeX{}
for typesetting your references. For formatting the bibliography
according to Springer's standard (for mathematics, physical
sciences, and computer science), please use the bibliography style
file \verb|splncs04.bst| that comes with the \texttt{llncs} document class.
You simply need to add \verb|\bibliographystyle{splncs04}| to your
document. DOIs should be provided in the doi field of your .bib
database. {\sc Bib}\TeX{} will then automatically add them to your
references.

\DescribeMacro{\doi}If you do not use {\sc Bib}\TeX{}, you can
include a DOI with the \verb|\doi| command:
\begin{verbatim}
\doi{<DOI>}
\end{verbatim}
The DOI will be expanded to the URL \verb|https://doi.org/<DOI>| in
accordance with the CrossRef guidelines.

\section{Obsolete Class Options}\label{SecClassOptions}
The \texttt{llncs} document class contains several class options that have
become obsolete over the years. We only mention them for
completeness:
\begin{itemize}
\item \DescribeMacro{orivec}The \texttt{llncs} document class changes the formatting of
vectors coded with \verb|\vec| to boldface italics. If you
absolutely need the original {\LaTeX} design for vectors, i.e. an
arrow above the related variable, you can restore it with the
\verb|orivec| option.
\item \DescribeMacro{envcountsame}All theorem-like environments
share one counter, i.e. Theorem 1, Lemma 2, Corollary 3, etc.
\item \DescribeMacro{envcountreset}All theorem-like environments
are numbered per section, i.e. the related counters are reset to 1
in every section.
\item \DescribeMacro{envcountsect}All theorem-like environments
are numbered per section, and the section number added to the
individual counter, i.e. Theorem 1.2, Lemma 2.2, etc.
\item \DescribeMacro{openbib}This option produces the ``open'' bibliography style, in which each block starts
on a new line, and succeeding lines in a block are indented by
\verb|\bibindent|.
\item \DescribeMacro{oribibl}This option restores the original
{\LaTeX} definitions for the bibliography and the \verb|\cite|
mechanism that some {\sc Bib}\TeX{} applications rely on.
\end{itemize}
%
%
\end{document}
