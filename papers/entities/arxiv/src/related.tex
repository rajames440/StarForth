% Section 2: Related Work
% Short, surgical - show awareness without picking fights

\section{Related Work}

\subsection{Adaptive Runtime Systems}

Self-tuning runtime environments have been explored across multiple domains. The Jikes RVM \cite{alpern2000jalapeno} demonstrated adaptive compilation with dynamic method selection. The HotSpot JVM employs tiered compilation and speculative optimization based on runtime profiling \cite{paleczny2001hotspot}. These systems optimize for common-case performance through feedback-driven specialization, though their convergence properties and steady-state characteristics are not typically characterized quantitatively.

PyPy's tracing JIT \cite{bolz2009tracing} uses execution traces to guide optimization, exhibiting adaptive behavior through guard-based speculation. SPUR \cite{bebenita2010spur} extended this model with staged compilation. Both demonstrate that adaptive mechanisms can improve performance, though behavioral stability under sustained workloads remains an open research question.

\subsection{Control-Theoretic Approaches}

Control theory has been applied to runtime resource management in systems such as AutoPilot \cite{diao2005using} and PCCP \cite{lu2008feedback}. These approaches model runtime behavior using feedback control loops with explicit stability analysis. Our work differs in observing emergent stability without explicit control-theoretic design, focusing on measurement rather than controller synthesis.

\subsection{Virtual Machine Instrumentation}

High-resolution VM instrumentation has enabled detailed performance analysis. Valgrind \cite{nethercote2007valgrind} provides fine-grained execution tracing. Pin \cite{luk2005pin} enables dynamic binary instrumentation. DynamoRIO \cite{bruening2004efficient} supports runtime code manipulation. These tools focus on observation and profiling; our instrumentation additionally feeds back into execution behavior.

\subsection{Operating System and Runtime Co-design}

Exokernel \cite{engler1995exokernel} explored application-level resource management. Singularity \cite{hunt2007singularity} demonstrated language-runtime-OS integration. L4Re/Fiasco.OC \cite{warg2009l4re} provides microkernel-based isolation with fine-grained resource control. These projects demonstrate that runtime-OS boundaries are negotiable; our work examines adaptive behavior within a single runtime layer.

Our work complements these prior efforts by documenting measurable steady-state properties in a minimal adaptive system. We focus on empirical characterization rather than performance optimization, stability guarantees, or architectural innovation.
\newpage