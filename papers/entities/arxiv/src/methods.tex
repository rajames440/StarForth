% Section 5: Experimental Methodology
% Reproducibility lives here - that sentence matters

\section{Experimental Methodology}

\subsection{Hardware Environment}

All experiments were conducted on a Beelink MINI S (AZW MINI S) compact workstation with the following specification:

\begin{itemize}
\item \textbf{Processor:} Intel N150 (4 cores)
\item \textbf{Memory:} 15.4 GiB RAM
\item \textbf{Graphics:} Intel Graphics (integrated)
\item \textbf{OS:} Ubuntu 24.10 (kernel 6.14.0-36-generic, 64-bit)
\item \textbf{Desktop:} KDE Plasma 6.3.4 on Wayland
\item \textbf{Frameworks:} KDE Frameworks 6.12.0, Qt 6.8.3
\end{itemize}

\textbf{Isolation measures.} To minimize measurement noise:
\begin{enumerate}
\item CPU frequency scaling disabled (governor set to \texttt{performance})
\item Process affinity set to dedicated core
\item All non-essential system services stopped
\item Network disabled during experiments
\item Disk I/O completed before measurement (all test files cached)
\end{enumerate}

\subsection{Software Configuration}

\textbf{VM implementation.} StarForth v0.9.1, compiled with:
\begin{itemize}
\item GCC 11.4.0
\item Optimization flags: \texttt{-O3 -march=native -flto}
\item Strict ANSI C99 compliance (\texttt{-std=c99 -pedantic})
\item All assertions disabled (\texttt{-DNDEBUG})
\end{itemize}

\textbf{Build targets.} Three build profiles were evaluated:
\begin{enumerate}
\item \textbf{standard:} Baseline optimizations, portable
\item \textbf{fast:} Architecture-specific optimizations, direct threading
\item \textbf{fastest:} Maximum optimizations (ASM, LTO, profile-guided)
\end{enumerate}

Primary results use the \textbf{fastest} target unless noted otherwise.

\textbf{Loop configurations.} Seven binary feedback loop toggles (L1–L7) yield $2^7 = 128$ configurations. Each configuration is encoded as a 7-bit string (e.g., \texttt{0000000} = all loops disabled, \texttt{1111111} = all loops enabled).

\subsection{Workload Classes}

Test workloads consist of deterministic FORTH programs exercising core VM functionality:

\begin{enumerate}
\item \textbf{Arithmetic stress:} Nested loops performing integer operations (addition, multiplication, modulo). Representative of compute-heavy tasks.
\item \textbf{Stack manipulation:} Sequences of \texttt{DUP}, \texttt{DROP}, \texttt{SWAP}, \texttt{ROT} operations. Tests stack handling efficiency.
\item \textbf{Control flow:} Nested \texttt{IF-THEN-ELSE} and \texttt{DO-LOOP} constructs. Evaluates branch prediction impact.
\item \textbf{Dictionary lookup:} Repeated word invocations with varying dictionary sizes. Tests lookup latency and cache effects.
\item \textbf{Factorial computation:} Recursive and iterative factorial implementations. Combined control flow and arithmetic.
\end{enumerate}

All workloads are deterministic: no randomness, no I/O, no user interaction. Execution order is fixed for each run.

\subsection{Replication Strategy}

\textbf{Design of Experiments (DoE).} Two experimental designs were conducted:

\begin{itemize}
\item \textbf{DoE-30:} 30 replicates per configuration, 128 configurations, 3840 total runs
\item \textbf{DoE-300:} 300 replicates per configuration, 128 configurations, 38400 total runs
\end{itemize}

DoE-30 provides initial screening; DoE-300 provides statistical power for variance analysis.

\textbf{Run sequencing.} Configurations were tested in pseudorandom order (fixed seed) to avoid systematic bias from time-of-day effects or hardware state drift. Within each configuration, replicates were executed sequentially with inter-run delays of 100 ms to allow thermal stabilization.

\textbf{Measurement protocol.} Each run captures:
\begin{enumerate}
\item Total execution time (wall-clock, nanosecond resolution)
\item Per-word execution heat distribution (64-bit counters)
\item Final K-statistic (computed from heat entropy and window metrics)
\item Metadata: timestamp, configuration ID, compiler flags, git commit hash
\end{enumerate}

\textbf{Reset conditions.} \emph{Each run starts from a clean VM state.} Specifically:
\begin{itemize}
\item VM memory zeroed
\item Dictionary rebuilt from source
\item All heat accumulators reset to zero
\item Rolling window cleared
\item Transition matrix zeroed
\item Cache state cold (no prefetching)
\end{itemize}

This ensures independence between runs: prior execution history does not contaminate subsequent measurements.

\subsection{Statistical Analysis}

\textbf{Descriptive statistics.} For each configuration, we compute:
\begin{itemize}
\item Mean runtime $\mu_t$, standard deviation $\sigma_t$, coefficient of variation CV
\item Mean K-statistic $\mu_K$, standard deviation $\sigma_K$
\item Percentiles: min, Q1, median, Q3, max
\end{itemize}

\textbf{Hypothesis testing.} Levene's test for variance homogeneity ($\alpha = 0.05$). ANOVA for main effects of loop configurations ($\alpha = 0.01$). Bonferroni correction applied for multiple comparisons.

\textbf{Goodness of fit.} James Law parameters ($\Lambda_{\text{eff}}$, $f_0$, $A_{\max}$, $W_{\text{decay}}$) estimated via nonlinear least squares. Model quality assessed via $R^2$, mean absolute deviation (MAD), and residual analysis.

\textbf{Reproducibility.} All analysis scripts (R 4.3.1), raw data (CSV format), and VM source code are archived in the supplementary materials repository. SHA-256 checksums provided for data integrity verification.