% Section 10: Conclusion
% One paragraph. Maybe two. Leave them thinking, not overwhelmed.

\section{Conclusion}

We have documented empirical observations of steady-state convergence in a minimal virtual machine with adaptive runtime mechanisms. The system implements seven configurable feedback loops operating concurrently without centralized control. Through 360 experimental runs spanning 128 runtime configurations, we observed deterministic convergence to reproducible execution states characterized by a dimensionless performance statistic $K$ with coefficient of variation below 1\%.

The statistic $K$ follows a predictable modulated inverse relationship with configuration parameters, described by an empirical law achieving $R^2 > 0.99$ goodness of fit. System behavior exhibits attractor-like characteristics including dual-mode states at resonance points and zero-variance rigid locking at anti-resonance. Empirical constants ($\Lambda_{\text{eff}} = 256$ bytes, $f_0 = 2/3$ cycles/window) remain consistent across deterministic workloads despite dynamic adaptation.

These results suggest that stable adaptive regimes may be achievable without centralized optimization. The existence of measurable invariants in a minimal implementation indicates such properties may be potentially non-accidental patterns rather than merely engineering artifacts. Whether these phenomena generalize to complex production runtimes, non-deterministic workloads, and diverse hardware platforms remains an open question warranting further investigation.

The observed steady states demonstrate that adaptive behavior and behavioral predictability need not be mutually exclusive. Runtime systems, operating system components, and distributed control mechanisms may exploit similar principles to achieve stable operation through local feedback without global coordination.