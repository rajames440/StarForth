% Abstract for arXiv Paper #1
% Clinical, factual, ~150-200 words
\begin{abstract}

We document empirical observations of steady-state behavior in a minimal virtual machine equipped with adaptive runtime mechanisms. The system implements seven configurable feedback loops that modify execution behavior based on runtime metrics including instruction heat, transition probabilities, and window-based history capture.

Using controlled workloads and high-resolution instrumentation, we measured runtime stability across 360 experimental runs spanning 128 distinct runtime configurations. Measurements include execution time, performance variance, and a dimensionless performance statistic $K$ derived from effective window utilization.

Results demonstrate deterministic convergence to stable execution regimes under fixed workload conditions. The statistic $K$ exhibits measurable invariance (coefficient of variation $<1\%$) across replicate runs and predictable modulation with window size. System behavior shows attractor-like characteristics, with configurations settling into reproducible states despite dynamic adaptation mechanisms.

These observations suggest that adaptive runtime systems can achieve measurable stability without sacrificing dynamic behavior. The existence of empirical invariants in a minimal implementation indicates that such properties may be potentially non-accidental patterns worthy of further investigation across diverse platforms.

\end{abstract}
\newpage
