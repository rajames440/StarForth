% =====================================================
% notation.tex
% Mathematical notation and symbol definitions
%
% This file can be:
% 1. Compiled standalone: pdflatex notation.tex
% 2. Included in another document: % =====================================================
% notation.tex
% Mathematical notation and symbol definitions
%
% This file can be:
% 1. Compiled standalone: pdflatex notation.tex
% 2. Included in another document: % =====================================================
% notation.tex
% Mathematical notation and symbol definitions
%
% This file can be:
% 1. Compiled standalone: pdflatex notation.tex
% 2. Included in another document: % =====================================================
% notation.tex
% Mathematical notation and symbol definitions
%
% This file can be:
% 1. Compiled standalone: pdflatex notation.tex
% 2. Included in another document: \input{formulas/notation.tex}
% =====================================================

\ifdefined\formulasincluded
\else
  \documentclass[11pt]{article}
  \usepackage{amsmath,amssymb,amsthm}
  \usepackage{booktabs}
  \usepackage[margin=1in]{geometry}
  \begin{document}
\fi

\subsection*{Mathematical Notation}

\subsubsection*{Primary Variables}

\begin{description}
\item[$K$] Performance statistic (dimensionless ratio $\Lambda_{\text{eff}} / W_{\text{actual}}$)
\item[$W$] Configured rolling window size (bytes)
\item[$W_{\text{actual}}$] Actual effective window size achieved by system (bytes)
\item[$\Lambda_{\text{eff}}$] Intrinsic characteristic wavelength (256 bytes)
\item[$H_e$] Execution heat of element $e$ (heat units)
\item[$H_{\text{total}}$] Sum of all execution heat values
\item[$P$] Performance metric (ns/word or cycles/instruction)
\item[$S$] Entropy of heat distribution (Shannon entropy, dimensionless)
\item[$\sigma^2$] Variance of timing measurements
\item[$t$] Time variable (seconds or heartbeat ticks)
\end{description}

\subsubsection*{Fundamental Constants}

\begin{description}
\item[$\lambda_0$] Intrinsic wavelength = 256 bytes (fundamental length scale)
\item[$f_0$] Natural frequency = $2/3$ cycles/window (standing wave frequency)
\item[$\varphi$] Golden ratio = $1.618\ldots$ (performance penalty ratio)
\item[$\kappa_0$] Window capacity constant (analogous to permeability $\mu_0$)
\item[$k_B$] Computational Boltzmann constant (heat-units/temperature)
\item[$\hbar_{\text{comp}}$] Computational "Planck constant" $\approx 0.05$
\end{description}

\subsubsection*{Derived Parameters}

\begin{description}
\item[$A(W)$] Amplitude envelope of standing wave modulation
\item[$A_{\max}$] Maximum amplitude $\approx 0.3$ (dimensionless)
\item[$W_{\text{decay}}$] Amplitude decay length scale $\approx 50000$ bytes
\item[$\phi$] Phase offset (radians)
\item[$\alpha$] Latency sensitivity parameter (1/heat-units)
\item[$n$] Integer quantization number: $W = n \times 256$ bytes
\end{description}

\subsubsection*{Runtime State Vector}

\[
\mathbf{\Psi}(W,t) = \begin{pmatrix}
K(W,t) \\
H_{\text{total}}(W,t) \\
P(W,t) \\
S(W,t) \\
\sigma^2(W,t)
\end{pmatrix}
\]

\subsubsection*{Operators}

\begin{description}
\item[$\nabla_W$] Configuration-space gradient (derivative with respect to window size)
\item[$\partial/\partial t$] Partial time derivative
\item[$\nabla_W \times$] Configuration-space curl
\item[$\nabla_W \cdot$] Configuration-space divergence
\item[$\nabla^2_W$] Configuration-space Laplacian
\item[$\langle \cdot \rangle_t$] Time average
\end{description}

\subsubsection*{Quantum-Analog Notation}

\begin{description}
\item[$|\psi\rangle$] State vector (Dirac notation)
\item[$|\text{locked}\rangle$] Locked attractor eigenstate ($K \approx \Lambda_{\text{eff}}/W$)
\item[$|\text{escaped}\rangle$] Escaped attractor eigenstate ($K \rightarrow 1.0$)
\item[$|\alpha|^2, |\beta|^2$] Occupation probabilities
\item[$\Delta E_{\text{eff}}$] Effective energy barrier (K-statistic units)
\item[$P_{\text{tunnel}}$] Tunneling probability
\end{description}

\subsubsection*{Validation Metrics}

\begin{description}
\item[CV] Coefficient of variation: $\text{CV} = \sigma / \mu$
\item[MAD] Mean absolute deviation
\item[$R^2$] Coefficient of determination (goodness of fit)
\item[$\chi^2$] Chi-squared statistic
\end{description}

\subsubsection*{Subscript/Superscript Conventions}

\begin{itemize}
\item Subscript $_{\text{eff}}$ indicates effective or measured quantity
\item Subscript $_{\text{baseline}}$ indicates baseline/unmodulated value
\item Subscript $_{\text{wave}}$ indicates wave-induced correction
\item Subscript $_{\text{comp}}$ indicates computational analog of physical constant
\item Subscript $_0$ indicates fundamental/natural scale
\item Subscript $_{i,j}$ indicates matrix indices (rows, columns)
\end{itemize}

\ifdefined\formulasincluded
\else
  \end{document}
\fi

% =====================================================

\ifdefined\formulasincluded
\else
  \documentclass[11pt]{article}
  \usepackage{amsmath,amssymb,amsthm}
  \usepackage{booktabs}
  \usepackage[margin=1in]{geometry}
  \begin{document}
\fi

\subsection*{Mathematical Notation}

\subsubsection*{Primary Variables}

\begin{description}
\item[$K$] Performance statistic (dimensionless ratio $\Lambda_{\text{eff}} / W_{\text{actual}}$)
\item[$W$] Configured rolling window size (bytes)
\item[$W_{\text{actual}}$] Actual effective window size achieved by system (bytes)
\item[$\Lambda_{\text{eff}}$] Intrinsic characteristic wavelength (256 bytes)
\item[$H_e$] Execution heat of element $e$ (heat units)
\item[$H_{\text{total}}$] Sum of all execution heat values
\item[$P$] Performance metric (ns/word or cycles/instruction)
\item[$S$] Entropy of heat distribution (Shannon entropy, dimensionless)
\item[$\sigma^2$] Variance of timing measurements
\item[$t$] Time variable (seconds or heartbeat ticks)
\end{description}

\subsubsection*{Fundamental Constants}

\begin{description}
\item[$\lambda_0$] Intrinsic wavelength = 256 bytes (fundamental length scale)
\item[$f_0$] Natural frequency = $2/3$ cycles/window (standing wave frequency)
\item[$\varphi$] Golden ratio = $1.618\ldots$ (performance penalty ratio)
\item[$\kappa_0$] Window capacity constant (analogous to permeability $\mu_0$)
\item[$k_B$] Computational Boltzmann constant (heat-units/temperature)
\item[$\hbar_{\text{comp}}$] Computational "Planck constant" $\approx 0.05$
\end{description}

\subsubsection*{Derived Parameters}

\begin{description}
\item[$A(W)$] Amplitude envelope of standing wave modulation
\item[$A_{\max}$] Maximum amplitude $\approx 0.3$ (dimensionless)
\item[$W_{\text{decay}}$] Amplitude decay length scale $\approx 50000$ bytes
\item[$\phi$] Phase offset (radians)
\item[$\alpha$] Latency sensitivity parameter (1/heat-units)
\item[$n$] Integer quantization number: $W = n \times 256$ bytes
\end{description}

\subsubsection*{Runtime State Vector}

\[
\mathbf{\Psi}(W,t) = \begin{pmatrix}
K(W,t) \\
H_{\text{total}}(W,t) \\
P(W,t) \\
S(W,t) \\
\sigma^2(W,t)
\end{pmatrix}
\]

\subsubsection*{Operators}

\begin{description}
\item[$\nabla_W$] Configuration-space gradient (derivative with respect to window size)
\item[$\partial/\partial t$] Partial time derivative
\item[$\nabla_W \times$] Configuration-space curl
\item[$\nabla_W \cdot$] Configuration-space divergence
\item[$\nabla^2_W$] Configuration-space Laplacian
\item[$\langle \cdot \rangle_t$] Time average
\end{description}

\subsubsection*{Quantum-Analog Notation}

\begin{description}
\item[$|\psi\rangle$] State vector (Dirac notation)
\item[$|\text{locked}\rangle$] Locked attractor eigenstate ($K \approx \Lambda_{\text{eff}}/W$)
\item[$|\text{escaped}\rangle$] Escaped attractor eigenstate ($K \rightarrow 1.0$)
\item[$|\alpha|^2, |\beta|^2$] Occupation probabilities
\item[$\Delta E_{\text{eff}}$] Effective energy barrier (K-statistic units)
\item[$P_{\text{tunnel}}$] Tunneling probability
\end{description}

\subsubsection*{Validation Metrics}

\begin{description}
\item[CV] Coefficient of variation: $\text{CV} = \sigma / \mu$
\item[MAD] Mean absolute deviation
\item[$R^2$] Coefficient of determination (goodness of fit)
\item[$\chi^2$] Chi-squared statistic
\end{description}

\subsubsection*{Subscript/Superscript Conventions}

\begin{itemize}
\item Subscript $_{\text{eff}}$ indicates effective or measured quantity
\item Subscript $_{\text{baseline}}$ indicates baseline/unmodulated value
\item Subscript $_{\text{wave}}$ indicates wave-induced correction
\item Subscript $_{\text{comp}}$ indicates computational analog of physical constant
\item Subscript $_0$ indicates fundamental/natural scale
\item Subscript $_{i,j}$ indicates matrix indices (rows, columns)
\end{itemize}

\ifdefined\formulasincluded
\else
  \end{document}
\fi

% =====================================================

\ifdefined\formulasincluded
\else
  \documentclass[11pt]{article}
  \usepackage{amsmath,amssymb,amsthm}
  \usepackage{booktabs}
  \usepackage[margin=1in]{geometry}
  \begin{document}
\fi

\subsection*{Mathematical Notation}

\subsubsection*{Primary Variables}

\begin{description}
\item[$K$] Performance statistic (dimensionless ratio $\Lambda_{\text{eff}} / W_{\text{actual}}$)
\item[$W$] Configured rolling window size (bytes)
\item[$W_{\text{actual}}$] Actual effective window size achieved by system (bytes)
\item[$\Lambda_{\text{eff}}$] Intrinsic characteristic wavelength (256 bytes)
\item[$H_e$] Execution heat of element $e$ (heat units)
\item[$H_{\text{total}}$] Sum of all execution heat values
\item[$P$] Performance metric (ns/word or cycles/instruction)
\item[$S$] Entropy of heat distribution (Shannon entropy, dimensionless)
\item[$\sigma^2$] Variance of timing measurements
\item[$t$] Time variable (seconds or heartbeat ticks)
\end{description}

\subsubsection*{Fundamental Constants}

\begin{description}
\item[$\lambda_0$] Intrinsic wavelength = 256 bytes (fundamental length scale)
\item[$f_0$] Natural frequency = $2/3$ cycles/window (standing wave frequency)
\item[$\varphi$] Golden ratio = $1.618\ldots$ (performance penalty ratio)
\item[$\kappa_0$] Window capacity constant (analogous to permeability $\mu_0$)
\item[$k_B$] Computational Boltzmann constant (heat-units/temperature)
\item[$\hbar_{\text{comp}}$] Computational "Planck constant" $\approx 0.05$
\end{description}

\subsubsection*{Derived Parameters}

\begin{description}
\item[$A(W)$] Amplitude envelope of standing wave modulation
\item[$A_{\max}$] Maximum amplitude $\approx 0.3$ (dimensionless)
\item[$W_{\text{decay}}$] Amplitude decay length scale $\approx 50000$ bytes
\item[$\phi$] Phase offset (radians)
\item[$\alpha$] Latency sensitivity parameter (1/heat-units)
\item[$n$] Integer quantization number: $W = n \times 256$ bytes
\end{description}

\subsubsection*{Runtime State Vector}

\[
\mathbf{\Psi}(W,t) = \begin{pmatrix}
K(W,t) \\
H_{\text{total}}(W,t) \\
P(W,t) \\
S(W,t) \\
\sigma^2(W,t)
\end{pmatrix}
\]

\subsubsection*{Operators}

\begin{description}
\item[$\nabla_W$] Configuration-space gradient (derivative with respect to window size)
\item[$\partial/\partial t$] Partial time derivative
\item[$\nabla_W \times$] Configuration-space curl
\item[$\nabla_W \cdot$] Configuration-space divergence
\item[$\nabla^2_W$] Configuration-space Laplacian
\item[$\langle \cdot \rangle_t$] Time average
\end{description}

\subsubsection*{Quantum-Analog Notation}

\begin{description}
\item[$|\psi\rangle$] State vector (Dirac notation)
\item[$|\text{locked}\rangle$] Locked attractor eigenstate ($K \approx \Lambda_{\text{eff}}/W$)
\item[$|\text{escaped}\rangle$] Escaped attractor eigenstate ($K \rightarrow 1.0$)
\item[$|\alpha|^2, |\beta|^2$] Occupation probabilities
\item[$\Delta E_{\text{eff}}$] Effective energy barrier (K-statistic units)
\item[$P_{\text{tunnel}}$] Tunneling probability
\end{description}

\subsubsection*{Validation Metrics}

\begin{description}
\item[CV] Coefficient of variation: $\text{CV} = \sigma / \mu$
\item[MAD] Mean absolute deviation
\item[$R^2$] Coefficient of determination (goodness of fit)
\item[$\chi^2$] Chi-squared statistic
\end{description}

\subsubsection*{Subscript/Superscript Conventions}

\begin{itemize}
\item Subscript $_{\text{eff}}$ indicates effective or measured quantity
\item Subscript $_{\text{baseline}}$ indicates baseline/unmodulated value
\item Subscript $_{\text{wave}}$ indicates wave-induced correction
\item Subscript $_{\text{comp}}$ indicates computational analog of physical constant
\item Subscript $_0$ indicates fundamental/natural scale
\item Subscript $_{i,j}$ indicates matrix indices (rows, columns)
\end{itemize}

\ifdefined\formulasincluded
\else
  \end{document}
\fi

% =====================================================

\ifdefined\formulasincluded
\else
  \documentclass[11pt]{article}
  \usepackage{amsmath,amssymb,amsthm}
  \usepackage{booktabs}
  \usepackage[margin=1in]{geometry}
  \begin{document}
\fi

\subsection*{Mathematical Notation}

\subsubsection*{Primary Variables}

\begin{description}
\item[$K$] Performance statistic (dimensionless ratio $\Lambda_{\text{eff}} / W_{\text{actual}}$)
\item[$W$] Configured rolling window size (bytes)
\item[$W_{\text{actual}}$] Actual effective window size achieved by system (bytes)
\item[$\Lambda_{\text{eff}}$] Intrinsic characteristic wavelength (256 bytes)
\item[$H_e$] Execution heat of element $e$ (heat units)
\item[$H_{\text{total}}$] Sum of all execution heat values
\item[$P$] Performance metric (ns/word or cycles/instruction)
\item[$S$] Entropy of heat distribution (Shannon entropy, dimensionless)
\item[$\sigma^2$] Variance of timing measurements
\item[$t$] Time variable (seconds or heartbeat ticks)
\end{description}

\subsubsection*{Fundamental Constants}

\begin{description}
\item[$\lambda_0$] Intrinsic wavelength = 256 bytes (fundamental length scale)
\item[$f_0$] Natural frequency = $2/3$ cycles/window (standing wave frequency)
\item[$\varphi$] Golden ratio = $1.618\ldots$ (performance penalty ratio)
\item[$\kappa_0$] Window capacity constant (analogous to permeability $\mu_0$)
\item[$k_B$] Computational Boltzmann constant (heat-units/temperature)
\item[$\hbar_{\text{comp}}$] Computational "Planck constant" $\approx 0.05$
\end{description}

\subsubsection*{Derived Parameters}

\begin{description}
\item[$A(W)$] Amplitude envelope of standing wave modulation
\item[$A_{\max}$] Maximum amplitude $\approx 0.3$ (dimensionless)
\item[$W_{\text{decay}}$] Amplitude decay length scale $\approx 50000$ bytes
\item[$\phi$] Phase offset (radians)
\item[$\alpha$] Latency sensitivity parameter (1/heat-units)
\item[$n$] Integer quantization number: $W = n \times 256$ bytes
\end{description}

\subsubsection*{Runtime State Vector}

\[
\mathbf{\Psi}(W,t) = \begin{pmatrix}
K(W,t) \\
H_{\text{total}}(W,t) \\
P(W,t) \\
S(W,t) \\
\sigma^2(W,t)
\end{pmatrix}
\]

\subsubsection*{Operators}

\begin{description}
\item[$\nabla_W$] Configuration-space gradient (derivative with respect to window size)
\item[$\partial/\partial t$] Partial time derivative
\item[$\nabla_W \times$] Configuration-space curl
\item[$\nabla_W \cdot$] Configuration-space divergence
\item[$\nabla^2_W$] Configuration-space Laplacian
\item[$\langle \cdot \rangle_t$] Time average
\end{description}

\subsubsection*{Quantum-Analog Notation}

\begin{description}
\item[$|\psi\rangle$] State vector (Dirac notation)
\item[$|\text{locked}\rangle$] Locked attractor eigenstate ($K \approx \Lambda_{\text{eff}}/W$)
\item[$|\text{escaped}\rangle$] Escaped attractor eigenstate ($K \rightarrow 1.0$)
\item[$|\alpha|^2, |\beta|^2$] Occupation probabilities
\item[$\Delta E_{\text{eff}}$] Effective energy barrier (K-statistic units)
\item[$P_{\text{tunnel}}$] Tunneling probability
\end{description}

\subsubsection*{Validation Metrics}

\begin{description}
\item[CV] Coefficient of variation: $\text{CV} = \sigma / \mu$
\item[MAD] Mean absolute deviation
\item[$R^2$] Coefficient of determination (goodness of fit)
\item[$\chi^2$] Chi-squared statistic
\end{description}

\subsubsection*{Subscript/Superscript Conventions}

\begin{itemize}
\item Subscript $_{\text{eff}}$ indicates effective or measured quantity
\item Subscript $_{\text{baseline}}$ indicates baseline/unmodulated value
\item Subscript $_{\text{wave}}$ indicates wave-induced correction
\item Subscript $_{\text{comp}}$ indicates computational analog of physical constant
\item Subscript $_0$ indicates fundamental/natural scale
\item Subscript $_{i,j}$ indicates matrix indices (rows, columns)
\end{itemize}

\ifdefined\formulasincluded
\else
  \end{document}
\fi
